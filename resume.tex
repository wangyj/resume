% Enable hyperlinks
\setupinteraction
  [state=start,
  style=,
  color=,
  contrastcolor=]

% make chapter, section bookmarks visible when opening document
\placebookmarks[chapter, section, subsection, subsubsection, subsubsubsection, subsubsubsubsection][chapter, section]
\setupinteractionscreen[option=bookmark]

\setuppagenumbering[location={footer,middle}]
\setupbackend[export=yes]
\setupstructure[state=start,method=auto]

% use microtypography
\definefontfeature[default][default][script=latn, protrusion=quality, expansion=quality, itlc=yes, textitalics=yes, onum=yes, pnum=yes]
\definefontfeature[smallcaps][script=latn, protrusion=quality, expansion=quality, smcp=yes, onum=yes, pnum=yes]
\setupalign[hz,hanging]
\setupitaliccorrection[global, always]

\setupbodyfontenvironment[default][em=italic] % use italic as em, not slanted

\definefallbackfamily[mainface][rm][CMU Serif][preset=range:greek, force=yes]
\definefontfamily[mainface][rm][Latin Modern Roman]
\definefontfamily[mainface][mm][Latin Modern Math]
\definefontfamily[mainface][ss][Latin Modern Sans]
\definefontfamily[mainface][tt][Latin Modern Typewriter][features=none]
\setupbodyfont[mainface]

\setupwhitespace[medium]

\setuphead[chapter]            [style=\tfd\setupinterlinespace,header=empty]
\setuphead[section]            [style=\tfc\setupinterlinespace]
\setuphead[subsection]         [style=\tfb\setupinterlinespace]
\setuphead[subsubsection]      [style=\bf]
\setuphead[subsubsubsection]   [style=\sc]
\setuphead[subsubsubsubsection][style=\it]

\setuphead[chapter, section, subsection, subsubsection, subsubsubsection, subsubsubsubsection][number=no]

\definedescription
  [description]
  [headstyle=bold, style=normal, location=hanging, width=broad, margin=1cm, alternative=hanging]

\setupitemize[autointro]    % prevent orphan list intro
\setupitemize[indentnext=no]

\setupfloat[figure][default={here,nonumber}]
\setupfloat[table][default={here,nonumber}]

\setupxtable[frame=off]
\setupxtable[head][topframe=on,bottomframe=on]
\setupxtable[body][]
\setupxtable[foot][bottomframe=on]


\starttext

\section[title={王永杰},reference={王永杰}]

\subsection[title={个人信息},reference={个人信息}]

\startitemize[packed]
\item
  姓名: 王永杰
\item
  Mobile: 138-114-00444
\item
  Email: yongjiemail@gmail.com
\stopitemize

\subsection[title={基本状况},reference={基本状况}]

2013年加入京东,任前台产品研发部技术专家、架构委员会主任架构师、手机京东开放平台负责人,专家委员会委员、前端技术委员会委员、共享技术部首席架构师等,是最早一批投身
Android 研究和开发的工程师。

目前重点投入 Flutter
等跨端技术研究及技术中台建设,致力于多端融合技术研究完成移动产品研发全生命周期工具化、自动化、系统化,提升研发效率和质量,降低业务开发门槛。

国防科大学士,北交大硕士,首届 GMTC 特邀嘉宾,第二、第三届 GMTC
优秀出品人&讲师。曾在盛大创新院等互联网公司担任高级研究员、资深架构师,是一个移动互联网的老兵。

\subsection[title={工作经历},reference={工作经历}]

\subsubsection[title={{\bf T10 架构师} @ 京东零售
共享技术部},reference={t10-架构师-京东零售-共享技术部}]

{\em 2013.6 - 至今}

\subsubsubsection[title={共享技术部架构师,前端通道委员会委员},reference={共享技术部架构师前端通道委员会委员}]

\startitemize[packed]
\item
  2018年-至今 技术与数据中心 移动技术中台 & 零售云 TPaaS
\item
  2016年-2018年 前台产品研发部 手机京东平台化
\item
  2014年-2016年 无线业务部 移动技术架构演进
\item
  2013年-2014年 移动研发部 创新业务开发
\stopitemize

\subsubsection[title={{\bf 高级研究员} @ 盛大创新院
多媒体主题院},reference={高级研究员-盛大创新院-多媒体主题院}]

\subsubsubsection[title={微酷Android负责人(Android开发共两人,项目团队共12人)},reference={微酷android负责人android开发共两人项目团队共12人}]

{\em 2012.3 - 2013.5}

\subsubsubsection[title={获得荣誉},reference={获得荣誉}]

\startitemize[packed]
\item
  2012年度,项目组荣获\quotation{最佳项目团队}奖。
\stopitemize

\subsubsubsection[title={酷6工作
{\em 2012.9-2013.5}},reference={酷6工作-2012.9-2013.5}]

\startitemize[packed]
\item
  参与移动团队长远目标规划和短期计划制定。
\item
  制定Android团队计划,设计产品与模块架构。
\item
  重新设计应用框架,偏重模块化,规范化 ,并使用新框架实现酷6微电影应用。
\item
  完成酷6视频新应用的\quotation{离线下载}模块。
\item
  开发和维护酷6微电影和ChannelV。
\item
  预研自有MediaPlayer视频播放引擎。
\item
  设计和开发酷6拍客。
\stopitemize

\subsubsubsection[title={创新院工作
{\em 2012.3-2012.9}},reference={创新院工作-2012.3-2012.9}]

\startitemize[packed]
\item
  完成微酷1.3,1.4和1.5三个版本的开发。
\item
  针对1.x存在问题,并根据2.x全新设计对Android版应用进行重新设计,模块化,框架搭建,代码规范化。
\item
  完成微酷各版本级定制版自动化开发版日构建和release版发布。
\item
  完成行为统计iAnalytics SDK及Demo开发,并协助集成至微酷和智能相册。
\item
  完成MediaRecorder SDK及Demo开发,并协助以插件形式集成至有你。
\item
  支援智能相册,解决重要bug,发现更深问题并解决。
\item
  协助Real2Virtual解决重要bug,如内存泄漏问题等。
\stopitemize

\subsubsection[title={{\bf 研发高级经理} @
北京乐投科技有限公司},reference={研发高级经理-北京乐投科技有限公司}]

\subsubsubsection[title={Android原生应用和Framework组负责人},reference={android原生应用和framework组负责人}]

{\em 2011.12 - 2012.3}

\subsubsubsection[title={主要工作},reference={主要工作}]

\startitemize[packed]
\item
  参与产品规划,设计及定义。
\item
  负责framework feature开发和bug fix等。
\item
  负责原生应用如Gallery, Music等开发和bug fix等。
\item
  负责Multimedia feature开发和bug fix等。
\item
  负责wowSearch MoviePlayer原型开发。
\item
  负责reader pad的Home原型开发等。
\stopitemize

\subsubsection[title={{\bf 软件架构师&技术经理} @
北京赛科世纪数码科技有限公司},reference={软件架构师技术经理-北京赛科世纪数码科技有限公司}]

\subsubsubsection[title={智能机顶盒软件架构师,Framework, Multimedia,
VOIP,SCM等六个team负责人},reference={智能机顶盒软件架构师framework-multimedia-voipscm等六个team负责人}]

{\em 2010.5 - 2011.12}

\subsubsubsection[title={获得荣誉},reference={获得荣誉-1}]

\startitemize[packed]
\item
  2010年度,项目组被评为赛科世纪\quotation{飞虎群英奖}。
\item
  2010年度,个人被评为赛科世纪\quotation{爱迪生创新奖}。
\stopitemize

\subsubsubsection[title={研发方向工作},reference={研发方向工作}]

\startitemize[packed]
\item
  初期担任软件架构师,主要负责:
  \startitemize[packed]
  \item
    负责产品规划和定义,制定年度目标。
  \item
    组建Android团队,由1人发展至7人。
  \item
    协助制板和移植工作,6月正式启动,7月移植初步成功。
  \stopitemize
\item
  中期兼任Multimedia团队team leader,团队由3人至7人。
  \startitemize[packed]
  \item
    多媒体团队工作组织和安排。
  \item
    完成基于OpenCore的AVI和MKV扩展。
  \item
    完成H.264等格式的硬解对接和兼容。
  \item
    完成MediaScanner扫描机制的修改,适应多个USB设备。
  \item
    完成VideoPlayer应用的开发。
  \item
    完成VideoPhone的移植和完善。
  \stopitemize
\item
  后续兼任Framework team leader,小团队共16人
  \startitemize[packed]
  \item
    MultiMedia方向4人,主要完成基于ffmpeg的StageFright插件等工作。
  \item
    VidePhone方向2人,主要负责VideoPhone的开发和维护,SIP协议,硬解码和硬编码,同时和客户的VideoPhone进行对接。
  \item
    Network方向2人,主要完成Wifi,Bluetooth和LAN的硬件适配和功能完善,实现手机遥控器的Service和Client。
  \item
    SCM
    方向3人,主要负责配置管理服务搭建和维护工作,比如分支的创建,版本的发布,流程梳理等。
  \item
    应用方向,3人,开发VideoPlayer3D,Music,Gallery3D,VideoPhone
    App,升级功能包括增量升级。
  \item
    GUI Framework方向,2人,主要完成framework功能开发和bug修改等。
  \stopitemize
\item
  Q3 主要担任软件三部(共约40人)技术经理,负责Android方向技术把握和支持。
  \startitemize[packed]
  \item
    主要包括Framework功能开发和bug修改。
  \item
    疑难问题解决如游戏移植闪烁问题和USB丢失文件等问题。
  \stopitemize
\stopitemize

\subsubsubsection[title={SCM 方向 -
为研发服务,节省开发时间,提升工作效率,减少人力工作。},reference={scm-方向---为研发服务节省开发时间提升工作效率减少人力工作}]

\subsubsection[title={{\bf 资深软件工程师} @ 播思通讯
GUI-Framework项目组},reference={资深软件工程师-播思通讯-gui-framework项目组}]

{\em 2008.10 - 2010.5}

\subsubsubsection[title={主要工作},reference={主要工作-1}]

\startitemize[packed]
\item
  负责Android输入法设计,开发,技术支持,bug fix以及性能优化等。
\item
  负责 GUI-Framework 文字相关功能扩展开发,维护以及技术支持等;
\item
  负责 GUI-Framework Widget 扩展、开发、维护和技术支持等;
\stopitemize

\subsubsubsection[title={主要成果},reference={主要成果}]

\startitemize[packed]
\item
  完成基于Android1.0的OMS1.0和FBW1.2平台12键拼音输入法。
\item
  完成基于Android1.0的OMS1.0和FBW1.2平台全键盘拼音输入法。
\item
  完成AutoText功能并merge到基于Android1.5的OMS1.5平台。
\item
  完成double click手势并提高用户体验和易用性。
\item
  参与三个Copy&Paste方案设计讨论和改进,并完成功能实现和易用性提高,最终设计进入OMS2.0发布。
\item
  开发和维护GUI-framework
  Widget如TextView,ScrollView,DynamicLayout等。
\item
  fix GUI-framework相关bug等。
\stopitemize

\subsubsection[title={{\bf 高级软件开发工程师} @ 灵图软件
技术预研项目组&数据检查系统(QCS)项目组},reference={高级软件开发工程师-灵图软件-技术预研项目组数据检查系统qcs项目组}]

{\em 2006.12 - 2008.10}

\subsubsubsection[title={获得荣誉},reference={获得荣誉-2}]

\startitemize[packed]
\item
  2007年度,项目组被评为灵图公司\quotation{年度卓越团队}。
\item
  2007年度,个人被评为灵图公司\quotation{年度创新之星}。 \#\#\#\#
  主要工作
\item
  数据检查系统一期(QCS),主要负责拓扑、形态类检查项开发及相关预研工作;
\item
  技术预研,主要负责GIS方向开源技术如GEOS等,为其他开发人员提供技术支持;
\item
  数据方向工具开发, 为数据生产提供支持;
\item
  数据检查系统二期(QCS2),
  主要负责系统重构优化设计和管理及相关预研工作等;
\item
  技术预研项目组,主要负责软件开发和GIS方向新技术的学习和研究,基础库的开发
  和维护以及部门技术方向的培训。
\stopitemize

\subsubsubsection[title={主要成果},reference={主要成果-1}]

\startitemize[packed]
\item
  熟练掌握诸多GIS开源技术如GDAL/OGR, GEOS, PROJ.4, MITAB, BOOST
  GRAPH等, 在项目开发中广泛应用,大大提高了开发效率并降低了开发难度;
\item
  提炼公用算法形成代码库或动态库,为部门做技术积累,减少类似工具开发的重复
  工作量,提供效率;
\item
  实现了很多原本用MapBasic在技术上无法实现的工具,如引导点生成工具;
\item
  在技术上对部门工具和项目开发进行支持,解决许多技术难点;
\item
  完成QCS2开发、单元测试、测试等整体框架,测试由原来逐项测试,手工比对结果
  提升到自动化批量测试的方式,减少了测试人员的工作量,大大提高了测试的效率;
\item
  完成了lt_matcher等基础库的开发,以支持工具、系统项目的开发;
\item
  进行技术培训,提升部门同事的开发能力和学习兴趣;
\stopitemize

\subsection[title={教育背景},reference={教育背景}]

\subsubsection[title={工学硕士 @
北京交通大学},reference={工学硕士-北京交通大学}]

\subsubsubsection[title={计算机与信息技术学院 信息科学研究所
信号与信息处理专业},reference={计算机与信息技术学院-信息科学研究所-信号与信息处理专业}]

{\em 2004.9 - 2007.1}

\subsubsubsection[title={发表论文},reference={发表论文}]

\startitemize[packed]
\item
  A Reversible Watermark Scheme Combined with Hash Function and Lossless
  Compression, Lecture Notes in Computer Science, Volume 3684/2005, pp:
  1168-1174,SCIE检索,第一作者
\item
  可以自恢复和篡改定位的可逆数字水印,哈尔滨工业大学学报,Vol. 38(Sup.),
  2006, pp: 791-794, EI检索,第一作者
\stopitemize

\subsubsubsection[title={开发实习},reference={开发实习}]

\startitemize[packed]
\item
  嵌入式设备Nand Flash 编程器开发项目等。
\item
  在北京神鹰广宇科技有限责任公司任C++程序设计讲师。
\stopitemize

\subsubsection[title={工学学士 @
国防科学技术大学},reference={工学学士-国防科学技术大学}]

\subsubsubsection[title={机电工程与自动化学院
自动化专业,},reference={机电工程与自动化学院-自动化专业}]

{\em 2000.9 - 2004.7}

\subsubsubsection[title={获得荣誉},reference={获得荣誉-3}]

\startitemize[packed]
\item
  2002 - 2003学年,被评为{\em 校优秀学员},位列专业{\em 第一名}。
\item
  2001、2003、2004学年三次荣获曾宪梓奖学金。
\stopitemize

\stoptext
