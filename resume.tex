\startmode[*mkii]
  \enableregime[utf-8]  
  \setupcolors[state=start]
\stopmode

% Enable hyperlinks
\setupinteraction[state=start, color=middleblue]

\setuppapersize [letter][letter]
\setuplayout    [width=middle,  backspace=1.5in, cutspace=1.5in,
                 height=middle, topspace=0.75in, bottomspace=0.75in]

\setuppagenumbering[location={footer,center}]

\setupbodyfont[11pt]

\setupwhitespace[medium]

\setuphead[chapter]      [style=\tfd]
\setuphead[section]      [style=\tfc]
\setuphead[subsection]   [style=\tfb]
\setuphead[subsubsection][style=\bf]

\setuphead[chapter, section, subsection, subsubsection][number=no]

\definedescription
  [description]
  [headstyle=bold, style=normal, location=hanging, width=broad, margin=1cm]

\setupitemize[autointro]    % prevent orphan list intro
\setupitemize[indentnext=no]

\setupfloat[figure][default={here,nonumber}]
\setupfloat[table][default={here,nonumber}]

\setupthinrules[width=15em] % width of horizontal rules

\setupdelimitedtext
  [blockquote]
  [before={\blank[medium]},
   after={\blank[medium]},
   indentnext=no,
  ]


\starttext

\subsection[个人信息]{个人信息}

\startitemize[packed]
\item
  姓名: 王永杰
\item
  Mobile: 138-114-00444/170-9014-9189/136-0124-0624
\item
  Email&Gtalk&MSN&QQ: yongjiemail@gmail.com
\stopitemize

\subsection[基本状况]{基本状况}

硕士学位,近八年工作经验,十年软件开发项目经验,六年+Android平台开发经验,近五年
Android架构和团队管理经验。19个月OPhone平台,19个月Android智能机顶盒,涉及
Multimedia,Framework,输入法,原生应用,微酷和酷6,京东彩票等应用。Android平台
团队建设、技术体系和研发环境部署经验丰富,熟练掌握各种SCM工具搭建和配置。

\subsection[能力概括]{能力概括}

\startitemize[packed]
\item
  十年+C/C++,六年+Java,熟练掌握Makefile和Shell,基本掌握Python和Php等脚本语言。
\item
  六年+Android开发经验,近五年架构和管理经验,涉及Framework, MultiMedia,
  原生应用 ,输入法, 微酷和酷6,京东彩票,CTS测试等。
\item
  SCM 经验丰富,搭建Android开发环境,Review,wiki,Bug Tracker以及Daily
  Build等。
\item
  熟练掌握Object Oriented,设计模式,重构以及单元测试等;
\item
  熟练使用Arch/Debian/Ubuntu等类Unix操作系统,VIM,
  Eclipse等各种开发工具,正则表达式等;
\item
  Win平台项目开发经验丰富, 熟练使用makefile, VC2005, VC6.0等IDE开发环境
\item
  熟练掌握GDAL/OGR, GEOS, 了解JTS, JCS, RoadMatcher等开源技术;
\item
  掌握扎实的计算机基础,数据结构,操作系统等,代码风格良好,质量高,逻辑严谨。
\item
  通过了英语四六级考试,具备良好的听说读写能力;
\item
  具有良好的团队协作精神和沟通表达能力,勇于并乐于接受挑战和探索尝试。
\item
  善于克服困难,分析和解决问题。
\stopitemize

\subsection[工作经历]{工作经历}

\subsubsection[软件架构师-京东商城-移动研发部]{{\bf 软件架构师} @
京东商城 移动研发部}

\subsubsubsection[京东android预研组负责人京东架构委员会委员京东评标专家]{京东Android预研组负责人,京东架构委员会委员,京东评标专家}

{\em 2013.6 - now}

\subsubsubsection[主要工作]{主要工作:}

\startitemize[packed]
\item
  完成京东彩票内嵌版一期,二期,三期,四期需求,设计,开发,测试并成功上线。
\item
  正在进行京东彩票内嵌版五期和独立客户端开发。
\item
  完成移动晒单项目开发
\item
  完成移动语音搜索项目开发
\item
  完成类apptimize的A/B测试工具原理研究和demo开发。
\item
  完成网络api基础模块架构设计封装和彩票api封装。
\item
  完成Git, Gerrit等工具和服务部署,完成daily build持续集成。
\item
  完成自动化单元测试,和dailybuild集成,自动运行和报告。
\item
  完成架构升级插件化项目Eclipse插件工具开发。
\item
  完成部门内分享:提高研发效率和产品质量之工具篇
\item
  解决京东商城客户端,闪购客户端部分疑难bug和优化等工作。
\item
  分析京东客户端图片加载慢的问题,提出分析结论和解决办法。
\stopitemize

\subsubsection[高级研究员-盛大创新院-多媒体主题院]{{\bf 高级研究员} @
盛大创新院 多媒体主题院}

\subsubsubsection[微酷android负责人android开发共两人项目团队共12人]{微酷Android负责人(Android开发共两人,项目团队共12人)}

{\em 2012.3 - 2013.5}

\subsubsubsection[获得荣誉]{获得荣誉}

\startitemize[packed]
\item
  2012年度,项目组荣获\quotation{最佳项目团队}奖。
\stopitemize

\subsubsubsection[酷6工作-2012.9-2013.5]{酷6工作 {\em 2012.9-2013.5}}

\startitemize[packed]
\item
  参与移动团队长远目标规划和短期计划制定。
\item
  制定Android团队计划,设计产品与模块架构。
\item
  重新设计应用框架,偏重模块化,规范化 ,并使用新框架实现酷6微电影应用。
\item
  完成酷6视频新应用的\quotation{离线下载}模块。
\item
  开发和维护酷6微电影和ChannelV。
\item
  预研自有MediaPlayer视频播放引擎。
\item
  设计和开发酷6拍客。
\stopitemize

\subsubsubsection[创新院工作-2012.3-2012.9]{创新院工作
{\em 2012.3-2012.9}}

\startitemize[packed]
\item
  完成微酷1.3,1.4和1.5三个版本的开发。
\item
  针对1.x存在问题,并根据2.x全新设计对Android版应用进行重新设计,模块化,框架搭建,代码规范化。
\item
  完成微酷各版本级定制版自动化开发版日构建和release版发布。
\item
  完成行为统计iAnalytics SDK及Demo开发,并协助集成至微酷和智能相册。
\item
  完成MediaRecorder SDK及Demo开发,并协助以插件形式集成至有你。
\item
  支援智能相册,解决重要bug,发现更深问题并解决。
\item
  协助Real2Virtual解决重要bug,如内存泄漏问题等。
\stopitemize

\subsubsection[研发高级经理-北京乐投科技有限公司]{{\bf 研发高级经理} @
北京乐投科技有限公司}

\subsubsubsection[android原生应用和framework组负责人]{Android原生应用和Framework组负责人}

{\em 2011.12 - 2012.3}

\subsubsubsection[主要工作-1]{主要工作}

\startitemize[packed]
\item
  参与产品规划,设计及定义。
\item
  负责framework feature开发和bug fix等。
\item
  负责原生应用如Gallery, Music等开发和bug fix等。
\item
  负责Multimedia feature开发和bug fix等。
\item
  负责wowSearch MoviePlayer原型开发。
\item
  负责reader pad的Home原型开发等。
\stopitemize

\subsubsection[软件架构师技术经理-北京赛科世纪数码科技有限公司]{{\bf 软件架构师&技术经理}
@ 北京赛科世纪数码科技有限公司}

\subsubsubsection[智能机顶盒软件架构师framework-multimedia-voipscm等六个team负责人]{智能机顶盒软件架构师,Framework,
Multimedia, VOIP,SCM等六个team负责人}

{\em 2010.5 - 2011.12}

\subsubsubsection[获得荣誉-1]{获得荣誉}

\startitemize[packed]
\item
  2010年度,项目组被评为赛科世纪\quotation{飞虎群英奖}。
\item
  2010年度,个人被评为赛科世纪\quotation{爱迪生创新奖}。
\stopitemize

\subsubsubsection[研发方向工作]{研发方向工作}

\startitemize[packed]
\item
  初期担任软件架构师,主要负责:
  \startitemize[packed]
  \item
    负责产品规划和定义,制定年度目标。
  \item
    组建Android团队,由1人发展至7人。
  \item
    协助制板和移植工作,6月正式启动,7月移植初步成功。
  \stopitemize
\item
  中期兼任Multimedia团队team leader,团队由3人至7人。
  \startitemize[packed]
  \item
    多媒体团队工作组织和安排。
  \item
    完成基于OpenCore的AVI和MKV扩展。
  \item
    完成H.264等格式的硬解对接和兼容。
  \item
    完成MediaScanner扫描机制的修改,适应多个USB设备。
  \item
    完成VideoPlayer应用的开发。
  \item
    完成VideoPhone的移植和完善。
  \stopitemize
\item
  后续兼任Framework team leader,小团队共16人
  \startitemize[packed]
  \item
    MultiMedia方向4人,主要完成基于ffmpeg的StageFright插件等工作。
  \item
    VidePhone方向2人,主要负责VideoPhone的开发和维护,SIP协议,硬解码和硬编码,同时和客户的VideoPhone进行对接。
  \item
    Network方向2人,主要完成Wifi,Bluetooth和LAN的硬件适配和功能完善,实现手机遥控器的Service和Client。
  \item
    SCM
    方向3人,主要负责配置管理服务搭建和维护工作,比如分支的创建,版本的发布,流程梳理等。
  \item
    应用方向,3人,开发VideoPlayer3D,Music,Gallery3D,VideoPhone
    App,升级功能包括增量升级。
  \item
    GUI Framework方向,2人,主要完成framework功能开发和bug修改等。
  \stopitemize
\item
  Q3 主要担任软件三部(共约40人)技术经理,负责Android方向技术把握和支持。
  \startitemize[packed]
  \item
    主要包括Framework功能开发和bug修改。
  \item
    疑难问题解决如游戏移植闪烁问题和USB丢失文件等问题。
  \stopitemize
\stopitemize

\subsubsubsection[scm-方向---为研发服务节省开发时间提升工作效率减少人力工作]{SCM
方向 - 为研发服务,节省开发时间,提升工作效率,减少人力工作。}

\subsubsection[资深软件工程师-播思通讯-gui-framework项目组]{{\bf 资深软件工程师}
@ 播思通讯 GUI-Framework项目组}

{\em 2008.10 - 2010.5}

\subsubsubsection[主要工作-2]{主要工作}

\startitemize[packed]
\item
  负责Android输入法设计,开发,技术支持,bug fix以及性能优化等。
\item
  负责 GUI-Framework 文字相关功能扩展开发,维护以及技术支持等;
\item
  负责 GUI-Framework Widget 扩展、开发、维护和技术支持等;
\stopitemize

\subsubsubsection[主要成果]{主要成果}

\startitemize[packed]
\item
  完成基于Android1.0的OMS1.0和FBW1.2平台12键拼音输入法。
\item
  完成基于Android1.0的OMS1.0和FBW1.2平台全键盘拼音输入法。
\item
  完成AutoText功能并merge到基于Android1.5的OMS1.5平台。
\item
  完成double click手势并提高用户体验和易用性。
\item
  参与三个Copy&Paste方案设计讨论和改进,并完成功能实现和易用性提高,最终设计进入OMS2.0发布。
\item
  开发和维护GUI-framework
  Widget如TextView,ScrollView,DynamicLayout等。
\item
  fix GUI-framework相关bug等。
\stopitemize

\subsubsection[高级软件开发工程师-灵图软件-技术预研项目组数据检查系统qcs项目组]{{\bf 高级软件开发工程师}
@ 灵图软件 技术预研项目组&数据检查系统(QCS)项目组}

{\em 2006.12 - 2008.10}

\subsubsubsection[获得荣誉-2]{获得荣誉}

\startitemize[packed]
\item
  2007年度,项目组被评为灵图公司\quotation{年度卓越团队}。
\item
  2007年度,个人被评为灵图公司\quotation{年度创新之星}。
  \#\#\#\#主要工作
\item
  数据检查系统一期(QCS),主要负责拓扑、形态类检查项开发及相关预研工作;
\item
  技术预研,主要负责GIS方向开源技术如GEOS等,为其他开发人员提供技术支持;
\item
  数据方向工具开发, 为数据生产提供支持;
\item
  数据检查系统二期(QCS2),
  主要负责系统重构优化设计和管理及相关预研工作等;
\item
  技术预研项目组,主要负责软件开发和GIS方向新技术的学习和研究,基础库的开发
  和维护以及部门技术方向的培训。
\stopitemize

\subsubsubsection[主要成果-1]{主要成果}

\startitemize[packed]
\item
  熟练掌握诸多GIS开源技术如GDAL/OGR, GEOS, PROJ.4, MITAB, BOOST
  GRAPH等, 在项目开发中广泛应用,大大提高了开发效率并降低了开发难度;
\item
  提炼公用算法形成代码库或动态库,为部门做技术积累,减少类似工具开发的重复
  工作量,提供效率;
\item
  实现了很多原本用MapBasic在技术上无法实现的工具,如引导点生成工具;
\item
  在技术上对部门工具和项目开发进行支持,解决许多技术难点;
\item
  完成QCS2开发、单元测试、测试等整体框架,测试由原来逐项测试,手工比对结果
  提升到自动化批量测试的方式,减少了测试人员的工作量,大大提高了测试的效率;
\item
  完成了lt_matcher等基础库的开发,以支持工具、系统项目的开发;
\item
  进行技术培训,提升部门同事的开发能力和学习兴趣;
\stopitemize

\subsection[教育背景]{教育背景}

\subsubsection[工学硕士-北京交通大学]{工学硕士 @ 北京交通大学}

\subsubsubsection[计算机与信息技术学院-信息科学研究所-信号与信息处理专业]{计算机与信息技术学院
信息科学研究所 信号与信息处理专业}

{\em 2004.9 - 2007.1}

\subsubsubsection[发表论文]{发表论文}

\startitemize[packed]
\item
  A Reversible Watermark Scheme Combined with Hash Function and Lossless
  Compression, Lecture Notes in Computer Science, Volume 3684/2005, pp:
  1168-1174,SCIE检索,第一作者
\item
  可以自恢复和篡改定位的可逆数字水印,哈尔滨工业大学学报,Vol. 38(Sup.),
  2006, pp: 791-794, EI检索,第一作者
\stopitemize

\subsubsubsection[开发实习]{开发实习}

\startitemize[packed]
\item
  嵌入式设备Nand Flash 编程器开发项目等。
\item
  在北京神鹰广宇科技有限责任公司任C++程序设计讲师。
\stopitemize

\subsubsection[工学学士-国防科学技术大学]{工学学士 @ 国防科学技术大学}

\subsubsubsection[机电工程与自动化学院-自动化专业]{机电工程与自动化学院
自动化专业,}

{\em 2000.9 - 2004.7}

\subsubsubsection[获得荣誉-3]{获得荣誉}

\startitemize[packed]
\item
  2002 - 2003学年,被评为{\em 校优秀学员},位列专业{\em 第一名}。
\item
  2001、2003、2004学年三次荣获曾宪梓奖学金。
\stopitemize

\subsection[自我评价和爱好]{自我评价和爱好}

\startitemize[packed]
\item
  开朗乐观自信,积极主动,喜欢思考,见解独到;
\item
  爱问为什么,极具潜力和领悟力,有很好的分析和解决问题能力;
\item
  对学习工作异常忘我和投入,总是充满了激情与活力。
\item
  热爱乒乓球、篮球等球类运动,曾获计算机学院乒乓球比赛男子单打冠军和团体赛季军;
\item
  热爱大自然,喜欢音乐,摄影,户外,登山等。
\stopitemize

\stoptext
